\documentclass{article}
\usepackage[francais]{babel}
\usepackage[utf8]{inputenc}
\usepackage{xcolor}
\usepackage[pdftex]{graphicx}
\usepackage{listings}
\usepackage{amsmath}
\usepackage[a4paper,includeheadfoot,margin=2.54cm]{geometry}
\usepackage{amsfonts}
\usepackage{fancyhdr}
\usepackage{titling}
\usepackage{algorithm}
\usepackage{algpseudocode}
\usepackage{hyperref}

\pagestyle{fancy}
\fancyhf{}
\fancyhead[LE,RO]{\theauthor}
\fancyhead[RE,LO]{\thetitle}
\fancyfoot[CE,CO]{\leftmark}
\fancyfoot[LE,RO]{\thepage}

%Syntax coloring C
\definecolor{mGreen}{rgb}{0,0.6,0}
\definecolor{mGray}{rgb}{0.5,0.5,0.5}
\definecolor{mPurple}{rgb}{0.58,0,0.82}
\definecolor{backgroundColour}{rgb}{1,1,1}

\lstdefinestyle{CStyle}{,
    backgroundcolor=\color{backgroundColour},   
    commentstyle=\color{mPurple},
    keywordstyle=\color{mGreen},
	identifierstyle=\color{blue},
    numberstyle=\tiny\color{mGray},
    stringstyle=\color{orange},
    basicstyle=\footnotesize,
    breakatwhitespace=false,         
    breaklines=true,                 
    captionpos=b,                    
    keepspaces=true,                 
    numbers=left,                    
    numbersep=5pt,                  
    showspaces=false,                
    showstringspaces=false,
    showtabs=false,                  
    tabsize=2,
    language=C
}
\usepackage[thinlines]{easytable}

\title{Rapport de TP : Alignement optimal et détection de plagiat}
\author{Annie LIM, Quentin GARRIDO}
\date{6 janvier 2020}

\begin{document}

\maketitle
\tableofcontents
\pagebreak

\section{Introduction}


%==============================================================================
\section{Exercice 1}

Pour calculer le score d'un alignement optimal entre x et y, nous pouvons utiliser
l'algorithme de distance de Levenshtein, appelé distance d'édition (edit distance).
Nous voulons observer les différences entre deux textes. Cela revient à calculer 
leur score d'alignement, le coût des opérations nécessaires (deletion, insertion,
substitution) pour obtenir le même texte. Plus ce score est faible et plus les textes
sont similaires, et donc sujet au plagiat.
Le score optimal correspond au minimum entre les trois valeurs données par les opérations
deletion, insertion et substitution. \\
Algo...\\
Cet algorithme est bien de complexité $O(|n|x|m|)$.


%==============================================================================
\section{Exercice 2}

A partir de la matrice T telle que $T[i][j]$ est le score d'un alignement
optimal etre $x_{i}$ et $y_{j}$, nous pourrons retrouver les opérations
nécessaires à la solution optimale pour aligner les deux textes, afin de
construire les textes 1 et 2 modifiés alignés.\\
Le backtracking consiste à suivre le chemin minimum de la matrice T de
$T[n][m]$ jusqu'à $T[0][0]$.\\
Algo...\\
Cet algorithme est bien de complexité o(|x|+|y|).\\

%==============================================================================
\section{Exercice 3}


%==============================================================================
\section{Exercice 4}



%=============================================================================
\section{Annexe: Code source}
		\lstinputlisting[style=CStyle]{../TD2.c}

\end{document}
