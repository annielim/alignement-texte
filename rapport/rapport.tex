\documentclass{article}
\usepackage[francais]{babel}
\usepackage[utf8]{inputenc}
\usepackage{xcolor}
\usepackage[pdftex]{graphicx}
\usepackage{listings}
\usepackage{amsmath}
\usepackage[a4paper,includeheadfoot,margin=2.54cm]{geometry}
\usepackage{amsfonts}
\usepackage{fancyhdr}
\usepackage{titling}
\usepackage{algorithm}
\usepackage{algpseudocode}
\usepackage{hyperref}

\pagestyle{fancy}
\fancyhf{}
\fancyhead[LE,RO]{\theauthor}
\fancyhead[RE,LO]{\thetitle}
\fancyfoot[CE,CO]{\leftmark}
\fancyfoot[LE,RO]{\thepage}

%Syntax coloring C
\definecolor{mGreen}{rgb}{0,0.6,0}
\definecolor{mGray}{rgb}{0.5,0.5,0.5}
\definecolor{mPurple}{rgb}{0.58,0,0.82}
\definecolor{backgroundColour}{rgb}{1,1,1}

\lstdefinestyle{CStyle}{,
    backgroundcolor=\color{backgroundColour},   
    commentstyle=\color{mPurple},
    keywordstyle=\color{mGreen},
	identifierstyle=\color{blue},
    numberstyle=\tiny\color{mGray},
    stringstyle=\color{orange},
    basicstyle=\footnotesize,
    breakatwhitespace=false,         
    breaklines=true,                 
    captionpos=b,                    
    keepspaces=true,                 
    numbers=left,                    
    numbersep=5pt,                  
    showspaces=false,                
    showstringspaces=false,
    showtabs=false,                  
    tabsize=2,
    language=C
}
\usepackage[thinlines]{easytable}

\title{Rapport de TP : Alignement optimal et détection de plagiat}
\author{Annie LIM, Quentin GARRIDO}
\date{6 janvier 2020}

\begin{document}

\maketitle
\tableofcontents
\pagebreak

\section{Introduction}

Ce TP a pour but de concevoir un logiciel d'aide à la détection de plagiat.\\
Ce logiciel d'alignement de séquences affichera simultanément le texte que l'on pense être du plagiat avec le texte original, en mettant en avant les correspondances. Moins les textes diffèrent et plus les chances de détecter un plagiat sont grandes.\\

%==============================================================================
\section{Exercice 1}

Pour calculer le score d'un alignement optimal entre x et y, nous pouvons utiliser l'algorithme de distance de Levenshtein, appelé distance d'édition (edit distance).\\
Nous voulons observer les différences entre deux textes. Cela revient à calculer leur score d'alignement, le coût des opérations nécessaires (deletion, insertion, substitution) pour obtenir le même texte. Plus ce score est faible et plus les textes sont similaires, et donc sujet d'être un plagiat.\\
Le score optimal correspond au minimum entre les trois valeurs données par les opérations deletion, insertion et substitution. \\
Algo...\\
Cet algorithme est bien de complexité o(\(\lvert x\rvert\)x\(\lvert y\rvert\)).


%==============================================================================
\section{Exercice 2}

Soit une matrice T telle que T[i][j] est le score d'un alignement optimal entre $x_{i}$ et $y_{j}$ avec $x_{i}$ ry $y_{j}$ les préfixes de x et y de longueur i et j.\\
A partir de cette matrice, nous pourrons retrouver les opérations nécessaires à la solution optimale pour aligner les deux textes, afin de construire les textes 1 et 2 modifiés alignés.\\
Le backtracking consiste à suivre le chemin minimum de la matrice T de T[i][j] jusqu'à T[0][0].\\
Algo...\\
Cet algorithme est bien de complexité o(\(\lvert x\rvert\)+\(\lvert y\rvert\)).\\

%==============================================================================
\section{Exercice 3}


%==============================================================================
\section{Exercice 4}



%=============================================================================
\section{Annexe: Code source}
		\lstinputlisting[style=CStyle]{../TD2.c}

\end{document}
